%++++++++++++++++++++++++++++++++++++++++
% Don't modify this section unless you know what you're doing!
\documentclass[letterpaper,11pt]{article}
\usepackage{tabularx} % extra features for tabular environment
\usepackage{amsmath}  % improve math presentation
\usepackage{graphicx} % takes care of graphic including machinery
\usepackage[margin=1in,letterpaper]{geometry} % decreases margins
\usepackage{cite} % takes care of citations
\usepackage[final]{hyperref} % adds hyper links inside the generated pdf file
\hypersetup{
	colorlinks=true,       % false: boxed links; true: colored links
	linkcolor=blue,        % color of internal links
	citecolor=blue,        % color of links to bibliography
	filecolor=magenta,     % color of file links
	urlcolor=blue         
}
%++++++++++++++++++++++++++++++++++++++++


\begin{document}

\title{\textbf{STATS 202C: Project 2\\Exact sampling of the Ising/Potts Model with coupled Markov Chains}}	
\author{Eric Chuu (604406828)}
\date{\today}
\maketitle


\section*{Introduction}

We consider the Ising model in an $n \times n$ lattice with 4 nearest neighbors. The state $X$ is a binary image defined on the lattice $X$, each at site or pixel $s$ takes value in $\{ 0, 1 \}$. The model sis
\begin{equation}
	& \pi(X) = \frac{1}{Z} \exp \{ \beta \sum_{\left\langle s, t \right\langle} \mathbb{1} \left( X_s = X_t \right) \}
\end{equation} \label{eq:gibbs} % the label is used to reference the equation

We simulate 2 Markov Chains with the Gibbs sampler:
\begin{itemize}
	\item MC1 starts with all sites being 1 (white chain) and its state is denoted by $X^1$
	\item MC2 starts with all sites being 0 (black chain) and its state is denoted by $X^2$
\end{itemize}

At each step, the Gibbs sampler picks up a site $s$ in both images, and calculates the conditional probabilities, which only depends on its 4 nearest neighbors, denoted by $\partial s$ 
\begin{align*}
	& \pi \left( X_{s}^1 | X_{\partial s}^{1} \right) \qquad \pi \left( X_{s}^2 | X_{\partial s}^{2} \right)
\end{align*}

It updates the variables $X_s^1$ and $X_s^2$ according to the above two conditional probabilities, and shares the same random number $r = \mathrm{rand}[0,1]$. 



\section*{Probem 1}
Prove that $X_s^1 \geq X_s^2$, for all $s$, in any time. That is, the white chain is always above the black chain. \\

\noindent
\textbf{Solution} \\


\pagebreak

\section*{Problem 2}
We plot the two chain states, using their total sum $\sum_{s} X_s^1$ and $\sum_{s} X_s^2$ over the sweeps, and we can see the image when the two chains coalesce. We use the following values for $\beta$:
$$\beta = \{ 0.5, 0.65, 0.75, 0.83, 0.84, 0.85, 0.9, 1.0 \}$$




\section*{Problem 3}
We plot the curve of $\tau$ versus $\beta$ to see a critical slowing-down around 0.84. \\


\end{document}
